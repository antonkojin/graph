% !TEX root = ./facebook.tex
\documentclass[a4paper, 11pt]{article}
\usepackage[latin1]{inputenc}
%\usepackage[T1]{fontenc}
\usepackage[italian]{babel}
\usepackage{listings}
\usepackage{color}
\usepackage{hyperref}
\hypersetup{
  colorlinks,
  citecolor = black,
  filecolor = black,
  linkcolor = black,
  urlcolor = black
}

\definecolor{mygreen}{rgb}{0,0.6,0}
\definecolor{mygray}{rgb}{0.5,0.5,0.5}
\definecolor{mymauve}{rgb}{0.58,0,0.82}

\lstset{
  %language=C,
  breaklines=true,
  basicstyle=\footnotesize\ttfamily,
  showstringspaces=false,
  captionpos=b,
  brackgroundcolor=,
  numbers=left,
  numbersep=5pt,
  numberstyle=\tiny\color{mygray},
  frame=l,
  backgroundcolor=\color{white},
  commentstyle=\color{mygreen},
  keywordstyle=\color{blue},
  rulecolor=\color{black},
  stringstyle=\color{mymauve},
  tabsize=2
  %keywordstyle=\bfseries\color{green},
  %commentstyle=\itshape\color{purple},
  %identifierstyle=\color{blue},
  %stringstyle=\color{orange}
}

\begin{document}

\title{
  {Progetto Algoritmi e Strutture Dati} \\
  {Facebook}
}
\author{
  { Anton Kozhin }\\
  { 830515 }
}
\date{\today}
\maketitle
\tableofcontents
\section{Strutture dati e algoritmi}
Rappresento gli utenti come nodi di un grafo non orientato e
le relazioni di amicizia come archi del grafo.
Un albero binario di ricerca %albero 2-3 ???
per trovare un utente dato il suo
identificativo.
L'algoritmo per la ricerca di gruppo di amici � la ricerca di componenti connesse del grafo.
\subsection{Grafo Utenti}
Facebook � composto da utenti e la relazione di amicizia tra due utenti. Una naturale rappresentazione delle relazioni � il grafo. Quindi facebook verr� rappresentato da un grafo non orientato e la relazione di amicizia come l'arco tra due vertici del grafo.

Rappresento gli archi con una lista di adiacenza, supponendo che il grafo sia poco connesso, visto che non tutti sono amici di tutti. Ho bisogno di una struttura dati composta per l'arco (\lstinline{edge}), poich� voglio memorizzare anche l'anno in cui si crea l'amicizia.
\lstinputlisting[language=C, firstline=18, lastline=24]{../codice/graph.h}

\lstinputlisting[language=C, firstline=12, lastline=17]{../codice/graph.h}

\lstinputlisting[language=C, firstline=4, lastline=8]{../codice/graph.h}
\lstinline{bst} � un albero binario di ricerca.

Lo spazio occupato dal grafo �
$
  S_{grafo} = O(|V| + |E|)
$

\subsubsection{Inserimento Utente}
Inserire un nuovo utente equivale a inserire un nuovo vertice nel grafo.
\begin{lstlisting}[mathescape]
aggiungi_vertice($G = (V,E)$, $v$):
  se $v \notin V$ allora:
    $V := V \cup \{v\}$
    inserisci $v$ nell'albero binario di ricerca di $V$
  altrimenti:
    stampa errore
  return $G$
\end{lstlisting}
Una stima della complessita computazionale col criterio uniforme.
Per verificare se $v \notin V$ faccio una ricerca nell'albero, ci si impiega $O(\log_{2}n)$ nel caso medio, dove $n$ � il numero di vertici nel grafo.
Il tempo per aggiungere $v$ all'insieme dei vertici $V$ � costante, si tratta di appenderlo in testa alla lista dei vertici di $G$.
Il tempo per inserire $v$ nell'albero di ricerca � anche lui logarimico.
\[
T_{aggiungi\_vertice} = O(\log_{2}n) + O(1) + O(\log_{2}n) = O(\log_{2}n)
\]

\subsubsection{Inserimento amicizia}
La seguente procedura ha come argomenti un grafo $G$, gli identificativi di due utenti \lstinline{id_u1} e \lstinline{id_u2} e il \lstinline{valore} da assegnare all'arco che colleghera i due vertici relativi ai due utenti. \lstinline[mathescape]{bst($G$)} restituisce l'albero binario di ricerca di $G$. $L($v$)$ restituisce la lista di adiacenza del vertice v.
\begin{lstlisting}[mathescape]
inserisci_amicizia($G=(V,E)$, id$_1$, id$_2$, valore)
  v$_1$ := ricerca_utente( bst($G$), id$_1$ )
  v$_2$ := ricerca_utente( bst($G$), id$_2$ )
  list_append( $L($v$_1)$, v$_2$, valore )
  list_append( $L($v$_2)$, v$_1$, valore )
\end{lstlisting}
\lstinline{ricerca_utente} costa $O(\log_{2}n)$. \lstinline{list_append} ha costo $O(1)$, � un inserimento in testa a una lista.
\[
  T_{inserisci\_amicizia} = 2\times O(\log_{2}n) + 2\times O(1) = O(\log_{2}n)
\]
\subsection{Albero binario di ricerca Utenti}
Ho implementato l'albero binario di ricerca per cercare un utente dato il suo identificativo. Altrimenti avrei dovuto cercare l'utente nella lista dei vertici, col costo di $O(n)$.
Strutture alternative per la ricerca potrebbero essere: tabella hash, efficiente ma inadatta per insiemi di dimensione variabile dato che si dovrebbe ridimensionare la tabella; albero 2-3, pi� efficiente nel caso peggiore, ma richiederebbe uno spazio maggiore per allocare la struttura.
\lstinputlisting[language=C, firstline=4, lastline=7]{../codice/binary_search_tree.h}
Lo spazio occupato dall'albero binario di ricerca �
$
  S_{bst} = O(|V|)
$

\subsubsection{Ricerca utente}
L'operazione di ricerca di un utente si traduce in una ricerca del suo identificativo nell'albero binario di ricerca (d'ora in poi \lstinline{BST}).
L'operatore \lstinline{id}restituisce l'identificativo relativo all'utente rappresentato dal nodo. La funzione \lstinline{dato} restituisce il dato a cui punta il nodo, in questo caso l'utente.
Banalmente le funzioni \lstinline{figlio_destro} e \lstinline{figlio_sinistro} restituiscono rispettivamente il figlio destro o sinistro del nodo.
\begin{lstlisting}[mathescape]
ricerca_utente(root, id_utente)
  if root = NULL
    return NULL
  if id_utente = id(root)
    return dato(root)
  else if id_utente < id(root)
    return ricerca_utente(figlio_sinistro(root), id_utente)
  else
    return ricerca_utente(figlio_destro(root), id_utente)
\end{lstlisting}
Questa procedura ricorsiva trova un utente in tempo $O(h)$ dove $h = \log_{2}(n)$ � l'altezza dell'albero, se supponiamo che l'albero sia bilanciato, cio� gli utenti inseriti in ordine sparso.
Tuttavia, nel caso peggiore, cio� se gli utenti sono inseriti in ordine crescente o decrescente di identificativo, l'albero degenera in una lista concatenata, e quindi la ricerca impiega $O(|V|)$.
\[
  T_{ricerca\_utente}^{medio}(n) = O(\log_{2}n)
\]
\[
  T_{ricerca\_utente}^{peggiore}(n) = O(n)
\]
\subsubsection{Inserimento utente nell'albero}
Nell'ambito del progetto questa procedura � eseguita solo se l'utente non � gia presente. %Definisco \lstinline{&x} come l'indirizzo di memoria di \lstinline{x}.
\begin{lstlisting}[mathescape]
inserisci_nodo(root, elemento):
  if root = NULL
    root = elemento
  else if id(elemento) < id(root)
    if figlio_sinistro(root) = NULL
      aggiungi_sinistro(root, elemento)
    else
      inserisci_nodo( figlio_sinistro(root), elemento )
  else
    if figlio_destro(root) = NULL
      aggiungi_destro(root, elemento)
    else
      inserisci_nodo( figlio_destro(root), elemento )
\end{lstlisting}
Per questa procedura valgono le stesse considerazoni fatte per \lstinline{ricerca_utente} presentata sopra.
\[
  T_{inserisci\_nodo}^{medio}(n) = O(\log_{2}n)
\]
\[
  T_{inserisci\_nodo}^{peggiore}(n) = O(n)
\]

\subsection{Gruppi di amici}
Dato un grafo degli utenti $G$, trovare i gruppi di amici.
Dalle specifiche si evince che un gruppo di amici � una componente connesa di $G$. $L(v)$ restituisce la lista dei vertici adiacenti al vertice $v$. $M(v)$ restituisce \lstinline{true} o \lstinline{false} a seconda se il vertice � stato marcato o meno come visitato.
\begin{lstlisting}[mathescape]
componente = NULL  // lista concatenata vuota, globale

componenti_connesse( $G=(V,E)$ )
  componenti := NULL  // lista vuota
  for each $v\in V$
    $M(v)$ := false
  for each $v \in V$
    if $M(v)$ = false
      list_append( componenti, profondit�($v$) );
  return componenti

profondit�($v$)
  list_append( componente, $v$)
  $M(v)$ := true
  for each $w \in L(v)$
    if $M(w)$ = false
    profondit�($w$)
\end{lstlisting}
\lstinline{list_append(lista, elemento)} inserisce in testa alla \lstinline{lista} l'\lstinline{elemento} in tempo $O(1)$.
Si ricorda inoltre che il costo complessivo delle chiamate a \lstinline{profondit�} � la somma delle lunghezze delle liste di adiacenza e che $\displaystyle \sum_{v \in V}{|L(v)|} = 2 \times |E| = O(|E|)$.
\[
  T_{componenti\_connesse} = O(|V|) + \sum_{v \in V}{T_{profondit�}(v)} = O(|V|) + \sum_{v \in V}{|L(v)|} = O(|V| +|E|)
\]

Per trovare i gruppi di amici tale che ogni amicizia � stata instaurata a partire da un certo anno, modifico la procedur� \lstinline{profondit�} nel modo seguente.
\begin{lstlisting}[mathescape]
profondit�($v$, anno)
  list_append( componente, $v$)
  $M(v)$ := true
  for each $w \in L(v)$
    if $M(w)$ = false $\land$ val($v, w$) >= anno
    profondit�($w$)
\end{lstlisting}
Dove con \lstinline[mathescape]{val($v, w$)} si intende il valore associato all'arco $(v, w)$
Questa condizione aggiuntiva non altera la stima della complessit�  che rimane $O(|V| + |E|)$.

\section{Conclusioni}
Una stima complessiva dello spazio utilizzato dalle strutture dati e del tempo computazionale delle procedure.
\[
  S_{facebook} = O(|V| + |E|) + O(|V|) = O(|V| + |E|)
\]
\[
  T_{facebook} = O(\log_{2}|V|) + O(|V| + |E|) = O(|V| + |E|)
\]
\end{document}
