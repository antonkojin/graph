% !TEX root = ./facebook.tex
\documentclass{article}
\usepackage{listings}
\usepackage{color}
\lstset{
language=C
%keywordstyle=\bfseries\color{green},
%commentstyle=\itshape\color{purple},
%identifierstyle=\color{blue},
%stringstyle=\color{orange}
}

\begin{document}

\title{
  {Progetto Algoritmi e Strutture Dati} \\
  {Facebook}
}
\author{ Anton Kozhin }
\date{\today}
\maketitle

\section{Introduzione}
Il progetto consiste nel

\section{Strutture dati e algoritmi}
Ho deciso di rappresentare gli utenti come nodi di un grafo non orientato e
le relazioni di amicizia come archi del grafo.
Un albero binario di ricerca per trovare un utente dato il suo
identificativo.
L'algoritmo per la ricerca di gruppo di amici è la ricerca di componenti
connesse in un grafo.
\subsection{Albero binario di ricerca Utenti}
Un albero binario di ricerca degli utenti, ordinato in base all' identificativo.

\subsubsection{Ricerca utente}
\subsection{Grafo Utenti}
\subsubsection{Inserimento Utente}
\begin{lstlisting}
#include <stdio.h>
int main(void){
 // cose
}
\end{lstlisting}
\subsubsection{Inserimento amicizia}
\subsection{Gruppi di amici}
\subsubsection{Dopo anno X}

%-------------------------------------
\section{Conclusion}
\end{document}
