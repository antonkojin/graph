% !TEX root = ./facebook.tex
\documentclass[a4paper, 11pt]{article}
\usepackage[latin1]{inputenc}
%\usepackage[T1]{fontenc}
\usepackage[italian]{babel}
\usepackage{listings}
\usepackage{color}

\definecolor{mygreen}{rgb}{0,0.6,0}
\definecolor{mygray}{rgb}{0.5,0.5,0.5}
\definecolor{mymauve}{rgb}{0.58,0,0.82}

\lstset{
  language=C,
  breaklines=true,
  basicstyle=\footnotesize\ttfamily,
  showstringspaces=false,
  captionpos=b,
  brackgroundcolor=,
  numbers=left,
  numbersep=5pt,
  numberstyle=\tiny\color{mygray},
  frame=l,
  backgroundcolor=\color{white},
  commentstyle=\color{mygreen},
  keywordstyle=\color{blue},
  rulecolor=\color{black},
  stringstyle=\color{mymauve},
  tabsize=2
  %keywordstyle=\bfseries\color{green},
  %commentstyle=\itshape\color{purple},
  %identifierstyle=\color{blue},
  %stringstyle=\color{orange}
}

\begin{document}

\title{
  {Progetto Algoritmi e Strutture Dati} \\
  {Facebook}
}
\author{ Anton Kozhin }
\date{\today}
\maketitle

\section{Introduzione}
Il progetto consiste nel

\section{Strutture dati e algoritmi}
Ho deciso di rappresentare gli utenti come nodi di un grafo non orientato e
le relazioni di amicizia come archi del grafo.
Un albero binario di ricerca %albero 2-3 ???
per trovare un utente dato il suo
identificativo.
L'algoritmo per la ricerca di gruppo di amici � la ricerca di componenti connesse.
\subsection{Grafo Utenti}
Facebook � composto da utenti e la relazione di amicizia tra due utenti. Una naturale rappresentazione delle relazioni � il grafo. Quindi facebook verr�rappresentato da un grafo non orientato e la relazione di amicizia come l'arco tra due vertici del grafo.

Ho bisogno di una struttura dati composta per l'arco, poich� voglio memorizzare anche l'anno in cui si crea l'amicizia.
\lstinputlisting[firstline=18, lastline=22, caption=Arco]{../codice/graph.h}

\lstinputlisting[firstline=12, lastline=16, caption=Vertice]{../codice/graph.h}

\lstinputlisting[firstline=6, lastline=10, caption=Grafo]{../codice/graph.h}
\lstinline{bst} � un albero binario di ricerca.

\subsubsection{Inserimento Utente}

\subsection{Albero binario di ricerca Utenti}
Un albero binario di ricerca degli utenti, ordinato in base all' identificativo.

\subsubsection{Ricerca utente}
\begin{lstlisting}[caption=Code]
#include <stdio.h>
int main(void){
 // cose
}
\end{lstlisting}
\subsubsection{Inserimento amicizia}
\subsection{Gruppi di amici}
\subsubsection{Dopo anno X}

%-------------------------------------
\section{Conclusion}
\end{document}
